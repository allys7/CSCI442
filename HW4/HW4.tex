\documentclass[12pt,largemargins]{homework}

% TODO: replace these with your information
\newcommand{\hwname}{Ally Smith}
\newcommand{\hwemail}{CSCI442}
\newcommand{\hwtype}{Homework}
\newcommand{\hwnum}{4}
\newcommand{\hwclass}{}
\newcommand{\hwlecture}{Section A}
\newcommand{\hwsection}{}

\newcommand{\code}{\texttt}

\begin{document}
\maketitle

\question{Many current language specifications, such as for C and C++, are
    inadequate for multi-threaded programs. This can have an impact on
    compilers and the correctness of code, as this problem illustrates.
    Consider the following declarations and function definition:}

\begin{verbatim}
    int global_positives = 0;
    typedef struct list {
        struct list *next;
        double val;
    } * list;

    void count_positives(list l) {
        list p;
        for (p = l; p; p = p -> next)
        if (p -> val > 0.0)
        ++global_positives;
    }
    \end{verbatim}

Now consider the case in which threads A executes
\code{count\_positives(<listcontaining only negative values>)}
while thread B executes \code{++global\_positives}.

\begin{alphaparts}
    \questionpart{What does the function do?}

    The function loops through the provided linked list, and counts the
    number of positive elements in the list.

    \questionpart{The C language only addresses single-threaded execution.
        Does the use of two parallel threads create any problems or potential
        problems?}

    Once \code{count\_positives} finishes running, \code{global\_positives} may
    have a value other than the number of positive numbers in the linked list,
    causing unexpected behavior.
\end{alphaparts}

\clearpage
\question{Some existing optimizing compilers (including gcc, which tends
    to be relatively conservative) will `optimize' count\_positives to
    something
    similar to the following:}

\begin{verbatim}
        void count_positives(list l) { 
            list p; 
            register int r; 
            r = global_positives; 
            for (p = l; p; p = p -> next) 
                if (p -> val > 0.0) ++r; 
            global_positives = r; 
        }
    \end{verbatim}

What problem or potential problems occurs with this compiled version of the
program if threads A and B are executed concurrently?

If thread B starts and finishes after thread A, then the
\code{global\_positives} will be incremented.
However, if thread A starts before thread B, then thread B will not be able to
increment since \code{global\_positives} since it will be reassigned from the
register.

\clearpage
\question{In the discussion of ULTs versus KLTs, it was pointed out that a
    disadvantage of ULTs is that when a ULT executes a system call, not only is
    that thread blocked, but also all of the threads within the process are
    blocked. Why is that so?}

This happens because, a lot of the time, system calls are blocking calls,
which means when the OS gets a syscall, it must block the whole process since
it doesn't recognize user-level threads.

\clearpage
\question{Consider an environment in which there is a one-to-one mapping
    between user-level threads and kernel-level threads that allows one or more
    threads within a process to issue blocking system calls while other threads
    continue to run. Explain why this model can make multi-threaded programs
    run faster than their single-threaded counterparts on a uni-processor
    computer.}

This would allow for multi-threaded programs to make blocking syscalls that don't
block their entire process, and instead just block that single thread. This would
allow for other threads to run as the scheduler sees fit.

\end{document}