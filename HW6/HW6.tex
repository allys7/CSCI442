\documentclass[11pt,largemargins]{homework}

% TODO: replace these with your information
\newcommand{\hwname}{Ally Smith}
\newcommand{\hwemail}{Section B}
\newcommand{\hwtype}{Homework}
\newcommand{\hwnum}{6}
\newcommand{\hwclass}{}
\newcommand{\hwlecture}{}
\newcommand{\hwsection}{}

\usepackage{enumitem} % format enumerate items
\begin{document}
\maketitle

\question{Suppose that we have a multiprogrammed computer in which each job has
identical characteristics. In one computation period $T$ for a job, half the
time is spent in I/O and the other half in processor activity. Each job runs for
a total of $N$ periods. Assume that a simple round-robin scheduling is used, and
that I/O operations can overlap with processor operation. Define the following
quantities:}
\begin{itemize}[noitemsep,topsep=0pt]
    \item Turnaround time = actual time to complete a job
    \item Throughput = average number of jobs completed per time period $T$
    \item Processor utilization = percentage of time that the processor is
    active (not waiting)
\end{itemize}
Compute these quantities for one, two, and four simultaneous jobs, assuming that
the period $T$ is distributed in the following way: I/O first half, processor
second half.

\begin{alphaparts}
\questionpart \textbf{One process}\\
    \begin{tabular}{|c|c|c|c|c|c|c|c|c|c|c|}\hline
        \textbf{CPU} &    &    & P1 & P1 &    &    & P1 & P1 & \ldots\\\hline
        \textbf{I/O} & P1 & P1 &    &    & P1 & P1 &    &    & \ldots\\\hline
    \end{tabular}
    
    Turnaround time: $N$

    Throughput: $\frac{N}{T}$

    Utilization: $50\%$

\questionpart \textbf{Two processes}\\
    \begin{tabular}{|c|c|c|c|c|c|c|c|c|c|c|}\hline
        \textbf{CPU} & P2 & P2 & P1 & P1 & P2 & P2 & P1 & P1 & \ldots\\\hline
        \textbf{I/O} & P1 & P1 & P2 & P2 & P1 & P1 & P2 & P2 & \ldots\\\hline
    \end{tabular}

    Turnaround time: $N$

    Throughput: $\frac{N}{T}$

    Utilization: $100\%$

\setlength{\tabcolsep}{2.3pt}
\questionpart \textbf{Four processes}\\
    \begin{tabular}{|c|c|c|c|c|c|c|c|c|c|c|c|c|c|c|c|c|c|c|}\hline
        \textbf{CPU} & P4 & P4 & P1 & P1 & P2 & P2 & P3 & P3 & P4 & P4 & P1 & P1 & P2 & P2 & P3 & P3 & \ldots\\\hline
        \textbf{I/O} & P1 & P1 & P2 & P2 & P3 & P3 & P4 & P4 & P1 & P1 & P2 & P2 & P3 & P3 & P4 & P4 & \ldots\\\hline
    \end{tabular}

    Turnaround time: $2N$

    Throughput: $\frac{N}{T}$

    Utilization: $100\%$
\end{alphaparts}

\clearpage
\question{Consider the following workload:}

\begin{center}
\begin{tabular}{|c|c|c|}\hline
    \textbf{Process} & \textbf{Burst Time} & \textbf{Arrival Time} \\\hline
    $P_1$ & 10 ms & 0 ms \\\hline
    $P_2$ & 20 ms & 5 ms \\\hline
    $P_3$ & 50 ms & 10 ms \\\hline
    $P_4$ & 30 ms & 15 ms \\\hline
    $P_5$ & 10 ms & 16 ms \\\hline
\end{tabular}
\end{center}
Show the schedule using Shortest Remaining Time, Round Robin with time quantum
of 10 ms, and Round Robin with time quantum of 20 ms. In addition, for each
algorithm, state the turnaround and response times for each process.


\textbf{Shortest Remaining Time}

\begin{tabular}{|c|c|c|c|}\hline
    \textbf{Time} & \textbf{Occurrence} & \textbf{Running} & \textbf{Queue} \\\hline
    0 & $P_1$ starts & $P_1$ & \\\hline
    5 & $P_2$ arrives & $P_1$ & $P_2$ \\\hline
    10 & $P_1$ finishes & $P_2$ & \\\hline
    10 & $P_3$ arrives & $P_2$ & $P_3$ \\\hline
    15 & $P_4$ arrives & $P_2$ & $P_4, P_3$ \\\hline
    16 & $P_5$ arrives & $P_5$ & $P_2, P_4, P_3$ \\\hline
    26 & $P_5$ finishes & $P_2$ & $P_4, P_3$ \\\hline
    40 & $P_2$ finishes & $P_4$ & $P_3$ \\\hline
    70 & $P_4$ finishes & $P_3$ & \\\hline
    120 & $P_3$ finishes & & \\\hline
\end{tabular}

\clearpage
\textbf{Round Robin} - TS 10 ms

\begin{tabular}{|c|c|c|c|}\hline
    \textbf{Time} & \textbf{Occurrence} & \textbf{Running} & \textbf{Queue} \\\hline
    0 & $P_1$ starts & $P_1$ & \\\hline
    5 & $P_2$ arrives & $P_1$ & $P_2$ \\\hline
    10 & $P_1$ finishes & $P_2$ & \\\hline
    10 & $P_3$ arrives & $P_2$ & $P_3$ \\\hline
    15 & $P_4$ arrives & $P_2$ & $P_3, P_4$ \\\hline
    16 & $P_5$ arrives & $P_2$ & $P_3, P_4, P_5$ \\\hline
    20 & $P_2$ expires & $P_3$ & $P_4, P_5, P_2$ \\\hline
    30 & $P_3$ expires & $P_4$ & $P_5, P_2, P_3$ \\\hline
    40 & $P_4$ expires & $P_5$ & $P_2, P_3, P_4$ \\\hline
    50 & $P_5$ finishes & $P_2$ & $P_3, P_4$ \\\hline
    60 & $P_2$ finishes & $P_3$ & $P_4$ \\\hline
    70 & $P_3$ expires & $P_4$ & $P_3$ \\\hline
    80 & $P_2$ finishes & $P_3$ & $P_4$ \\\hline
    90 & $P_3$ expires & $P_4$ & $P_3$ \\\hline
    100 & $P_4$ finishes & $P_3$ & \\\hline
    110 & $P_3$ expires & $P_3$ & \\\hline
    120 & $P_3$ finishes & & \\\hline
\end{tabular}

% \clearpage
\textbf{Round Robin} - TS 20 ms

\begin{tabular}{|c|c|c|c|}\hline
    \textbf{Time} & \textbf{Occurrence} & \textbf{Running} & \textbf{Queue} \\\hline
    0 & $P_1$ starts & $P_1$ & \\\hline
    5 & $P_2$ arrives & $P_1$ & $P_2$ \\\hline
    10 & $P_1$ finishes & $P_2$ & \\\hline
    10 & $P_3$ arrives & $P_2$ & $P_3$ \\\hline
    15 & $P_4$ arrives & $P_2$ & $P_3, P_4$ \\\hline
    16 & $P_5$ arrives & $P_2$ & $P_3, P_4, P_5$ \\\hline
    30 & $P_2$ finishes & $P_3$ & $P_4, P_5$ \\\hline
    50 & $P_3$ expires & $P_4$ & $P_5, P_3$ \\\hline
    70 & $P_4$ expires & $P_5$ & $P_3, P_4$ \\\hline
    80 & $P_5$ finishes & $P_3$ & $P_4$ \\\hline
    100 & $P_3$ expires & $P_4$ & $P_3$ \\\hline
    110 & $P_4$ finishes & $P_3$ & \\\hline
    120 & $P_3$ finishes & & \\\hline 
\end{tabular}

\clearpage
\question{Which type of process is generally favored by a multi-level feedback
queueing scheduler --- a process-bound process or an I/O-bound process? Briefly
explain why.}

An I/O-bound process is favored by an MLFQ scheduling algorithm, as CPU-bound
processes are moved down in priority, while I/O-bound processes are moved to a
higher queue.
\end{document}