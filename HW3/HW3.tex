\documentclass[11pt,largemargins]{homework}
\usepackage{amsmath}

% TODO: replace these with your information
\newcommand{\hwname}{Ally Smith}
\newcommand{\hwemail}{Section A}
\newcommand{\hwtype}{Homework}
\newcommand{\hwnum}{3}
\newcommand{\hwclass}{CSCI442}
\newcommand{\hwlecture}{}
\newcommand{\hwsection}{A}

\begin{document}
\maketitle

% Your content
\question{Consider the following code:}

\begin{verbatim}
    char buf[100];
    int fd0 = open(“foo.txt”, O_RDONLY);
    int fd1 = open(“bar.txt”, O_WRONLY | O_CREAT | O_APPEND,
                    0666)
    printf(“Hola!\n”);
    dup2(fd1, STDOUT_FILENO);
    printf(“Como estas?\n”)
    dup2(fd0, STDIN_FILENO);
    for (;;) {
        ssize_t bytes_read = read(STDIN_FILENO,
                                buf, sizeof(buf));
        if (bytes_read == 0)
            return 0;
        ssize_t bytes_to_write = bytes_read;
        while (bytes_to_write > 0)
            bytes_to_write -= write(STDOUT_FILENO,
                buf + bytes_read – bytes_to_write,
                bytes_to_write);
    }
    \end{verbatim}

The program is invoked like so:

\begin{verbatim}
    $ cat foo.txt
    Muy bien!
    $ cat bar.txt
    Buenos dias!
    $ ./a.out
    \end{verbatim}

\begin{alphaparts}
    \questionpart{What was printed on standard output?}
    \begin{verbatim}
    Hola!
    Buenos dias!
    Como estas?
    Muy bien!
    \end{verbatim}

    \questionpart{What does foo.txt contain?}
    \begin{verbatim}
    Muy bien!
    \end{verbatim}

    \questionpart{What does bar.txt contain?}
    \begin{verbatim}
    Buenos dias!
    \end{verbatim}
\end{alphaparts}
\clearpage

\question{Consider the following code:}

\begin{verbatim}
    int pipefd[2];
    pipe(pipefd);
\end{verbatim}

Assume that the operating system assigns the file descriptor \texttt{3} for the
read end of the pipe, and the file descriptor \texttt{4} for the write end of
the pipe.

\begin{alphaparts}
    \questionpart{What is \texttt{pipefd[0]} after this operation is
        performed?}

    \texttt{3}

    \questionpart{What is \texttt{pipefd[1]} after this operation is
        performed?}

    \texttt{4}

\end{alphaparts}
\clearpage

\question{The following state transition table is a simplified model of process
    management, with the labels representing transitions between states of
    \texttt{READY}, \texttt{RUN}, \texttt{BLOCKED}, and \texttt{NONRESIDENT}.}

\begin{tabular}{|c|c|c|c|c|}\hline %chktex 44
    $\tfrac{to}{from}$ & \textbf{READY} & \textbf{RUN} & \textbf{BLOCKED} &
    \textbf{NONRESIDENT}
    \\\hline %chktex 44
    READY              & --             & 1            & --               & 5
    %chktex 8
    \\\hline %chktex 44
    RUN                & 2              & --           & 3                & --
    %chktex 8
    \\\hline %chktex 44
    BLOCKED            & 4              & --           & --               & 6
    %chktex 8
    \\\hline %chktex 44
\end{tabular}

\begin{arabicparts}
    \questionpart{\texttt{READY} $\Rightarrow$ \texttt{RUN}}

    A process could go through this transition if it is the next process in
    line to be executed by the operating system.

    \questionpart{\texttt{RUN} $\Rightarrow$ \texttt{READY}}

    This transition could occur if a process times out.

    \questionpart{\texttt{RUN} $\Rightarrow$ \texttt{BLOCKED}}

    This transition could occur if a process has to wait for an I/O
    operation.

    \questionpart{\texttt{BLOCKED} $\Rightarrow$ \texttt{READY}}

    A process could go through this transition if an I/O event occurs that
    a process was waiting on.

    \questionpart{\texttt{READY} $\Rightarrow$ \texttt{NONRESIDENT}}

    A process could go through this transition if it is suspended into
    secondary memory.

    \questionpart{\texttt{BLOCKED} $\Rightarrow$ \texttt{NONRESIDENT}}

    A process could go through this transition if it is suspended into
    secondary memory.
\end{arabicparts}
\clearpage

\question{Assume that at time 5 no system resources are being used except for
    the processor and memory. Now consider the following events:}

\begin{tabular}{r l}
    T5:  & P1 executes a command to read from disk unit 3.               \\
    T15: & P3's time slice expires.                                      \\
    T18: & P4 executes a command to write to disk unit 3.                \\
    T20: & P2 executes a command to read from disk unit 2.               \\
    T24: & P3 executes a command to write to disk unit 3.                \\
    T28: & P3 is swapped out.                                            \\
    T33: & An interrupt occurs from disk unit 2: P2's read is complete.  \\
    T36: & An interrupt occurs from disk unit 3: P1's read is complete.  \\
    T38: & P5 terminates.                                                \\
    T40: & An interrupt occurs from disk unit 3: P3's write is complete. \\
    T44: & P3 is swapped back in.                                        \\
    T48: & An interrupt occurs from disk unit 3: P4's write is complete. \\
\end{tabular}

For each time 25, 35, and 45, identify which state (or possible states)
each process is in. If a process is
blocked, further identify the event on which it is blocked.

\vspace{.5in}
\begin{tabular}{|c|c|c|c|}\hline %chktex 44
        & Time 25               & Time 35               & Time
    45
    \\\hline %chktex 44

    P1 & \texttt{BLOCKED}, DU3 & \texttt{BLOCKED}, DU3
        & \texttt{READY}
    \\\hline %chktex 44

    P2 & \texttt{BLOCKED}, DU2 & \texttt{READY}
        & \texttt{READY}
    \\\hline %chktex 44

    P3 & \texttt{BLOCKED}, DU3 & \texttt{NONRESIDENT}  &
    \texttt{NONRESIDENT}
    \\\hline %chktex 44

    P4 & \texttt{BLOCKED}, DU3 & \texttt{BLOCKED}, DU3 &
    \texttt{BLOCKED}, DU3
    \\\hline %chktex 44

    P5 & \texttt{RUN}          & \texttt{RUN}          &
    \texttt{EXIT}
    \\\hline %chktex 44
\end{tabular}

\end{document}